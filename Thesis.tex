# Bachelor Thesis

# 0. Cover

# 1. Status Quo (What is the Situation?)

# 2. The Problem (What is the Problem?)

# 3. The Solution (What do we do to sole this? How does it work?)

... All these problems could be worked around but this would leave us in a less than optimal situation.

Instead, I propose designing a domain specific language (DSL) in order to create a text based notation (or code) for the configuration of a Dialogflow agent, which I will refer to as “agent” from now on.

This DSL must fulfill a number of requirements to be usable:

- It needs to be significantly shorter than the JSON representation of an agent
- It needs to be more readable than the JSON representation of an agent.
- It needs to be automatically compiled and updated on the web.
- It needs to create smaller diffs (sum of changed lines in a file) than the JSON representation when making changes

If we can manage to do this, we´ll have solved the problem.

# 4. Why does it work? (skip for now)

# 5. When does it work?

The DSL should provide an improvement for your project if it fullfills any of the following conditions:
- You are working with a team of 3 or more developers.
- You are allready using git and are doing code reviews regularly.
- Your Voice Assistant is a core part of your system, that receives updates regularly. 
- You want to use continous integration with your voice Assistant.

# 6. Why does it work when it works?

# 7. Evaluation (How well does it work?)

# 8. What further work could be done?

# 9. Declaration of Independence!

# 10. Apendix
 - Code
 - Abstract in German


"Versioning for Parameterised system"
What are the differences between perivous technology and dialogflow configuration language.

Opposite Apporach to the same problewm:
We have a thing we want to work with, but it doesn't fint into version control. 
Their approach: Build new version control for this. 
My approach: Change the thing to fit into existing version control.

Instead of trying to impact the existing landscape of version cotrol with a competitor too git specifically for Dialogflow, make DIalogflow fit into Git.
 
http://www.ic.uff.br/~leomurta/papers/murta2008.pdf
https://dl.acm.org/citation.cfm?id=1109129
https://link.springer.com/chapter/10.1007/978-3-540-69073-3_31
https://scholar.google.com.br/scholar?hl=de&as_sdt=0%2C5&q=version+control+XML&btnG=
https://scholar.google.com.br/scholar?hl=de&as_sdt=0%2C5&q=MDE+version+control&btnG=


Sources:

https://scholar.google.de/scholar?hl=de&as_sdt=0%2C5&q=%22Dialogflow%22&btnG=
https://scholar.google.de/scholar?hl=de&as_sdt=0%2C5&q=git+version+control&btnG=&oq=Git+Versio
https://scholar.google.de/scholar?hl=de&as_sdt=0%2C5&q=%22Version+Control%22&btnG=


Likely Usefull:
Version Control with Git: Powerful Tools and Techniques for Collaborative ...
von Jon Loeliger, Matthew McCullough
Pages 2-4

Why Git for your organization
https://www.atlassian.com/git/tutorials/why-git



Facts:
more than 188M repos on Github alone in 2019:
https://api.github.com/repositories?since=188000000



Unlikely usefull:
Hands-On Chatbots and Conversational UI Development: Build chatbots and ...
von Srini Janarthanam
Pages 146ff

To Be looked at:

http://www0.cs.ucl.ac.uk/staff/a.finkelstein/fose/fdestublier.pdf

https://books.google.de/books?hl=de&lr=&id=8CrCXVZXLjcC&oi=fnd&pg=PT20&dq=domain+specific+language&ots=1pwxUbQWLv&sig=yxYGlnaQwt1lF_FhayL-6P2nMnM#v=onepage&q=domain%20specific%20language&f=false

https://www.sciencedirect.com/science/article/pii/S0164121200000893
