\chapter*{Abstract (Deutsch)}

Diese Bachelorarbeit widmet sich einem derzeit bestehenden Mangel an Versionskontrollsystemen für Sprach-Interfaces.
Anstatt aber ein neues Versionskontrollsystem für Sprach-Interfaces zu bauen, wird hier eine andere Lösung gewählt. 
Ein Sprach-Interface wird so adaptiert, dass es mit bestehenden Versionskontrollsystemen kompatibel ist.
Zu diesem Zweck wird eine domänenspezifische Sprache entwickelt, die ein Sprach-Interface auf eine textbasierte Art spezifiziert. 
Um diese Lösung zu demonstrieren, wurde die Sprach-Interface-Technologie Dialogflow gewählt, in Verbindung mit dem Versionskontrollsystem Git.
Für die Evaluation wurde ein Experiment mit 5 Testszenarien durchgeführt, in dem die domänenspzifische Sprache mit der existierenden Technologie verglichen wurde. Die Ergebnisse zeigen einen klaren Vorteil durch die Nutzung der DSL bei drei von vier zuvor spezifizierten Kriterien, ohne Verlust beim vierten Kriterium.