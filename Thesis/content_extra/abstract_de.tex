\chapter{Abstract (Deutsch)}

Diese Bachelorarbeit widmed sich einem derzeit bestehenden mangel an versionscontrollsystemen  für stimmliche benutzerschnittstellen.
Anstatt aber ein neues versionscontrollsystem für stimmliche benutzerschnittstellen zu bauen wird eine andere lösung gewählt. 
Eine stimmliche benutzerschnittstelle wird so adaptiert, dass sie mit bestehenden versionscontrollsystemen kompatibel ist.
Zu diesem  zweck wird eine domänenspezifische sprache entwickelt, die eine stimmliches benutzerschnittstelle auf eine textbasierten art spezifiziert. 
Um diese lösung zu demonstrieren wurde die sprach interface technologie Dialogflow gewählt, in verbindung mit dem versionscontrollsystem Git.
Für die Evaluation wurde ein experiment mit 5 test szenarien durgeführt, in dem die dömanenspzifische sprache mit der existierenden technologie verglichen wurde. Die ergebnisse zeigen einen klaren gewinn bei drei von vier zuvor spezifizierten kriterien, ohne verlust beim virten kriterium.
